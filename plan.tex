%%%%%%%%%%%%%%%%%%%%%%%%%%%%%%%%%%%%%%%%%%%%%%%%%%%%%%%%%%%%%%%%%%%%%%%%%%%%%%%%
%2345678901234567890123456789012345678901234567890123456789012345678901234567890
%        1         2         3         4         5         6         7         8

\documentclass[letterpaper, 10 pt, conference]{ieeeconf}  % Comment this line out
                                                          % if you need a4paper
%\documentclass[a4paper, 10pt, conference]{ieeeconf}      % Use this line for a4
                                                          % paper

\IEEEoverridecommandlockouts                              % This command is only
                                                          % needed if you want to
                                                          % use the \thanks command
                                                          \usepackage[utf8]{inputenc}
\usepackage[utf8]{inputenc}
\overrideIEEEmargins
% See the \addtolength command later in the file to balance the column lengths
% on the last page of the document



% The following packages can be found on http:\\www.ctan.org
%\usepackage{graphics} % for pdf, bitmapped graphics files
%\usepackage{epsfig} % for postscript graphics files
%\usepackage{mathptmx} % assumes new font selection scheme installed
%\usepackage{times} % assumes new font selection scheme installed
\usepackage{amsmath} % assumes amsmath package installed
\usepackage{amssymb}  % assumes amsmath package installed
\usepackage{fancyhdr}
\setlength{\headheight}{15.2pt}
\usepackage[]{algorithm2e}

\newtheorem{thm}{Theorem}
\newtheorem{lem}{Lemma}
\newtheorem{defi}{Definition}
\newtheorem{prop}{Proposition}

\title{\LARGE \bf
%modif
On the list coloring of 1-band buffering cellular graphs
}

%\author{ \parbox{3 in}{\centering Huibert Kwakernaak*
%         \thanks{*Use the $\backslash$thanks command to put information here}\\
%         Faculty of Electrical Engineering, Mathematics and Computer Science\\
%         University of Twente\\
%         7500 AE Enschede, The Netherlands\\
%         {\tt\small h.kwakernaak@autsubmit.com}}
%         \hspace*{ 0.5 in}
%         \parbox{3 in}{ \centering Pradeep Misra**
%         \thanks{**The footnote marks may be inserted manually}\\
%        Department of Electrical Engineering \\
%         Wright State University\\
%         Dayton, OH 45435, USA\\
%         {\tt\small pmisra@cs.wright.edu}}
%}

\author{Marine Collery$^{1}$ and Benjámin Martin Seregi$^{2}$% <-this % stops a space
% \thanks{*This work was not supported by any organization}% <-this % stops a space
\thanks{$^{1}$ student of Research Methodology and Scientific Writing Course at KTH Kista P1P22017. e-mail: collery@kth.se}%
\thanks{$^{2}$ student of Research Methodology and Scientific Writing Course at KTH Kista P1P22017. e-mail: seregi@kth.se}%
}

\begin{document}


\maketitle
\thispagestyle{fancy}
\fancyhf{}
\chead{KTH Royal Institute of Technology | II2202, Fall 2017, Period 1-2 | \today}

\begin{abstract}
The optimal channel allocation problem (CAP) in cellular networks is often formulated in a graph theoretic framework. One of its variants\---where each access point knows the list of its free channels\---is related the so-called list coloring problem. In spite of the fact that the list coloring problem is NP-complete for arbitrary graphs, we showed that there exists a polynomial time algorithm for 1-band buffering callular graphs; and as a corollary it turned out the choice number of such a graphs is at most 4. In addition, we carried out performance comparisons between the existing integer linear programming solution and our implementation.
\end{abstract}

%%%%%%%%%%%%%%%%%%%%%%%%%%%%%%%%%%%%%%%%%%%%%%%%%%%%%%%%%%%%%%%%%%%%%%%%%%%%%%%%
\section{Introduction}
%Describe the background for chosen area that is going to be investigated. Write a short description of the area that is going to be %investigated. It is a brief description of the necessary background knowledge of the problem area and for carrying out the project.

In telecommunication one of the most challenging problems is the efficient allocation of the available frequency. Since the available bandwidth is always limited (and expensive), the efficient utilization of the frequency spectrum is a major concern. Due to the fastest growing number of mobile Internet users, the optimal channel allocation in cellular networks and their variants have been heavily researched in recent years \cite{Audhya:2011:SCA:1988563.1988571}.

Several variants of the channel allocation problem (CAP) have been defined based on the different channel constraints that a particular service might require. One of them is the so-called co-channel constraint where the same channel is not allowed to be assigned to neighboring cells simultaneously. This problem have been formalized as a graph coloring problem by many authors \cite{1456167}. Unfortunately, graph coloring is a well-known NP-complete problem \cite{Kar72} and therefore we do not know if a polynomial time algorithm for co-channel constraint satisfaction exists. Therefore various heuristic algorithms have been developed, the list of methods includes genetic algorithms, neural networks, graph-based and other approaches as well \cite{Audhya:2011:SCA:1988563.1988571}.

Cellular network topologies usually admit certain geometric structure. The most common network topology is the hexagonal grid topology where each cell is represented by a regular hexagon (two cells are neighbors if they share a common boundary). In \cite{662943}, the authors exploited this special structure and proposed an algorithm that optimally solves the CAP in $k$-band buffering systems where $k=1,2$. Moreover, the algorithm has polynomial running time $O(p)$ where $p$ is the number of cells.

The authors of \cite{7248845} introduced a \textit{distinctly different} CAP from all the above mentioned problems. Assuming a $2$-band buffering hexagonal cell topology (the interference graph created from this topology is called cellular graph) where each cell has a fixed number of frequency channel (channels are either busy or free), they asked the following question: "\textit{What is smallest size of the set of free channels associated with the cells (nodes of the cellular graph) that can guarantee interference free channel assignment to all the nodes?}". This problem is related to one of the generalizations of the graph coloring problem, called list coloring. It turned out that the required number of free channels is between $8$ and $10$. In addition, two algorithms have been proposed to create an interference free assignment, that is, a list-coloring of the cellular graph. The first one is the integer linear programming formulation of the list coloring problem (and therefore it is not a polynomial algorithm), the second one is a heuristic linear time algorithm that is, according to their experiment, within 12\% of the optimal solutions.

In what follows, we formulate the same problem in $1$-band buffering systems, then we outline a possibly polynomial time algorithm that optimally solves the list coloring problem in cellular graphs.

\section{Don't know}

\begin{lem}\label{lem:degree-constraint}
Let $G$ be a cellular graph. Then $G$ has a node $v$ such that $\deg(v) \leq 3$.
\end{lem}

\begin{proof} Let us consider the planar drawing of the graph $G$ which corresponds to its hexagonal topology.
Let $v$ be any node of $G$ and we assign $(0,0)$ to this node. We assign coordinates to every node starting from node $v$ in the following way. If a node $w$ is to the north of node $v$ then the coordinates of node $w$ is equal to the coordinates of node $v$ plus $(0,1)$. We summarize this method in Table \ref{table:assignment} according to the cardinal directions.
\begin{table}[!h]
\centering
 \begin{tabular}{||c|c||} 
 \hline
 Direction & Change  \\ [0.5ex] 
 \hline\hline
 N & $(0,1)$  \\ 
 \hline
 NE & $(1,1)$ \\
 \hline
 SE & $(1,0)$ \\
 \hline
 S & $(0,-1)$ \\
 \hline
 SW & $(-1,-1)$ \\
 \hline
 NW & $(-1,0)$ \\
 \hline
 \end{tabular}
  \caption{Rules of the hexagonal coordinate system}\label{table:assignment}
\end{table}
By applying this method to every node, we get a coordinate system within our hexagonal topology. Since there are finitely many node in $V(G)$, we can select the maximal node by selecting the node with maximal $x$- and $y$-coordinate, that is, a node $w$ with the following property: if $(x_1,y_1)$ is the coordinates of $w$ and $v \neq w$ is another node with the coordinates $(x_2,y_2)$ then $x_1 \geqslant x_2$ and $y_1 > y_2$ are satisfied.

To obtain a contradiction, suppose that the maximal node $w$ has more than 3 neighbors. It follows then at least one if its neighbors is either to the north, northeast or southeast to $w$. However, this neighbor would violate the maximality of $w$. Therefore we proved that $\deg(w) \leq 3$.
\end{proof}
\begin{defi}
We say that the orientation of a digraph $G$ is $k$\textit{-bounded} if for each node $v \in V(G)$ its outdegree is bounded by $k$, that is, $\deg^+(v) \leqslant k$.
\end{defi}
\begin{lem}\label{lem:bounded-acyclic-orientation}
If $G$ is a cellular graph, then $G$ has a $3$-bounded acyclic orientation.
\end{lem}
\begin{proof} Since $G=(V,E)$ is a cellular graph, we can make use of Lemma \ref{lem:degree-constraint}, that is, let $v$ be a node of $G^{(0)}:=G$ with $\deg(v) \leq 3$. We construct a function $f\colon V(G) \to \mathbb{N}$ by setting $f(v) := 0$ and then remove the node $v$ from $G^{(0)}$ yielding the graph $G^{(1)}$. By repeating the same procedure - at step $i$th - we select a node $v$ in $G^{(i)}$ such that $\deg(v) \leqslant 3$ and we set $f(v):=i$. Then we remove $v$ from $G^{(i)}$ which yields the graph $V(G^{(i+1)}) := V(G^{(i)}) \setminus \lbrace v \rbrace$. It is trivial that the graphs $G^{(1)},\ldots,G^{(|V(G)|)}$ are all cellular therefore it was valid using Lemma \ref{lem:degree-constraint}.

We note that the procedure ends after $|V(G)|$ steps.

Now we construct a digraph $H=(V,A)$ from $G$ using the function $f$. Let $(u,v) \in E$ be an arbitrary edge. If $f(u) < f(v)$ then $A:=A\cup (u,v)$ otherwise $A:=A \cup (v,u)$. By repeating this procedure for all edges in $E$ we get an orientation of $G$, that is, the digraph $H$. We need to prove that $H$ is
\begin{enumerate}
\item acyclic and
\item $\deg^+(v) \leqslant 3$ holds for all $v \in V(G)$.
\end{enumerate}
To prove (1), we will obtain a contradiction by assuming that $H$ contains a directed cycle. Let $C$ be a directed cycle in $H$ with the nodes $C:=\lbrace v_1,v_2,\ldots,v_k \rbrace$ that are ordered such that $(v_i,v_{i+1 \bmod{k}}) \in E$ $(i=1,2,\ldots,k)$ which means that $f(v_1) < f(v_2) < \ldots < f(v_{k-1}) < f(v_k)$ but also $f(v_k) < f(v_1)$ which is impossible.

What is left to show is that $\deg^+(v) \leqslant 3$. At each step we select a node with no more than $3$ neighbors and those neighbors will be assigned a number that is greater than $f(v)$. Therefore the outdegree of $v$ cannot be larger than $3$.
\end{proof}

It is easy to see that the proof of this lemma can be transformed into a polynomial time algorithm. In what follows we outline the algorithm that is based on the proof of Lemma \ref{lem:bounded-acyclic-orientation} and then we prove that its running time is $O(|V(G)| + |E(G)|)$:
\begin{algorithm}\label{alg:find-acyclic-bounded-orientation}
 \KwData{A cellular graph $G = (V,E)$}
 \KwResult{An acyclic $3$-bounded orientation of $G$}
 $G^{(0)} := G$\;
 Initialize $f \colon V(G) \to \mathbb{N}$\;
 \For{$i\leftarrow 0$ \KwTo $|V(G)|-1$}{
  Select $v \in V(G)$ such that $\deg(v) \leqslant 3$\;
  $f(v):=i$\;
  $V(G^{(i+1)}):=V(G^{i}) \setminus \lbrace v \rbrace$\;
 }
 Initialize $H:=(V,A)$\;
 \ForAll{$e = (u,v) \in E(G)$}{
   \eIf{$f(u) < f(v)$} {
   	$A := A \cup (u,v)$\;
   }{
   $A := A \cup (v,u)$\;
   }   
 }
 \caption{Constructing an acyclic $3$-bounded orientation of a cellular graph}
\end{algorithm}
\begin{prop} Algorithm \ref{alg:find-acyclic-bounded-orientation} has a running time of $O(|V(G)| + |E(G)|)$.
\end{prop}
\begin{proof}
It is enough to prove that "Select $v \in V(G)$ such that $\deg(v) \leqslant 3$" can be done in $O(|V(G)|)$ since the rest is obvious. Let us initialize a queue $Q := \lbrace v \mid \deg(v) \leqslant  3\rbrace$ before we start running the algorithm. We pop a node $v$ from $G$ at every iteration and push new ones after the removal of $v$ if there are new nodes in $G$ that satisfy the degree condition. The pop and push methods can be implemented in constant time which completes the proof.
\end{proof}
\addtolength{\textheight}{-12cm}   % This command serves to balance the column lengths
                                  % on the last page of the document manually. It shortens
                                  % the textheight of the last page by a suitable amount.
                                  % This command does not take effect until the next page
                                  % so it should come on the page before the last. Make
                                  % sure that you do not shorten the textheight too much.

%%%%%%%%%%%%%%%%%%%%%%%%%%%%%%%%%%%%%%%%%%%%%%%%%%%%%%%%%%%%%%%%%%%%%%%

\bibliographystyle{IEEEtran}
\bibliography{ref.bib}
\end{document}